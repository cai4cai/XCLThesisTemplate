%%%%%%%%%%%%%%%%%%%%%%%%%%%%%%%%%%%%%%%%%%%%%%%%%
% Common tools
%%%%%%%%%%%%%%%%%%%%%%%%%%%%%%%%%%%%%%%%%%%%%%%%%
\usepackage{ifpdf}


%%%%%%%%%%%%%%%%%%%%%%%%%%%%%%%%%%%%%%%%%%%%%%%%%
% Math
%%%%%%%%%%%%%%%%%%%%%%%%%%%%%%%%%%%%%%%%%%%%%%%%%
\usepackage{amsmath} % load before txfonts
\usepackage{amssymb}


%%%%%%%%%%%%%%%%%%%%%%%%%%%%%%%%%%%%%%%%%%%%%%%%%
% Algorithms and theorems
%%%%%%%%%%%%%%%%%%%%%%%%%%%%%%%%%%%%%%%%%%%%%%%%%
%\usepackage[amsmath,thmmarks,hyperref]{ntheorem}

%\setlength{\theorempostskipamount}{15pt}

%\theoremstyle{break}
%\theoremheaderfont{\normalfont\bfseries}
%\theoremseparator{}
%\theorembodyfont{\normalfont}
%\theoremsymbol{\rule{1ex}{1ex}}
%\newtheorem{algorithm}{Algorithm}


%%%%%%%%%%%%%%%%%%%%%%%%%%%%%%%%%%%%%%%%%%%%%%%%%
% Fonts and Encoding
%%%%%%%%%%%%%%%%%%%%%%%%%%%%%%%%%%%%%%%%%%%%%%%%%
\usepackage[utf8]{inputenc}

% Use T1 font encoding
\usepackage[T1]{fontenc}

% Add a few more symbols: mu, bullet points, etc.
\usepackage{textcomp}


%%%%%%%%%%%%%%%%%%%%%%%%%%%%%%%%%%%%%%%%%%%%%%%%%
% Figures
%%%%%%%%%%%%%%%%%%%%%%%%%%%%%%%%%%%%%%%%%%%%%%%%%
%\usepackage{placeins} % for \FloatBarrier
%\usepackage[margin=10pt,font=small,labelfont=bf,tableposition=top]{caption}
%\usepackage[font=footnotesize]{subfig}
\usepackage{graphicx}


%%%%%%%%%%%%%%%%%%%%%%%%%%%%%%%%%%%%%%%%%%%%%%%%%
% Tables
%%%%%%%%%%%%%%%%%%%%%%%%%%%%%%%%%%%%%%%%%%%%%%%%%
% Correct spacing for top captions
%\usepackage{topcapt}
%\newcommand{\topcaption}{\caption}

% Tool to merge several rows
\usepackage{multirow}

% Tool to split a cell (upper left) with a diagonal line (slash)
%\usepackage{slashbox}
\usepackage{diagbox}

% Choose a smaller column separation
\setlength{\tabcolsep}{3pt}


%%%%%%%%%%%%%%%%%%%%%%%%%%%%%%%%%%%%%%%%%%%%%%%%%
% Definitions
%%%%%%%%%%%%%%%%%%%%%%%%%%%%%%%%%%%%%%%%%%%%%%%%%
\DeclareMathOperator{\Id}{Id}
\DeclareMathOperator{\Expec}{\mathbb{E}}
\DeclareMathOperator{\grad}{grad}

\newcommand{\argmax}{\operatornamewithlimits{arg\,max}}
\newcommand{\argmin}{\operatornamewithlimits{arg\,min}}
\DeclareMathOperator{\median}{median}

\DeclareMathOperator{\trace}{trace}
\DeclareMathOperator{\vect}{Vect}

\providecommand{\abs}[1]{\left\lvert#1\right\rvert}
\providecommand{\norm}[1]{\left\lVert#1\right\rVert}
\providecommand{\dotprod}[2]{\langle #1\,\vert\, #2 \rangle}
\providecommand{\dist}[2]{\operatorname{dist}\left( #1 , #2 \right)}

\providecommand{\liebracket}[2]{\left[ #1\,,\, #2 \right]}
\DeclareMathOperator{\ad}{ad}
\DeclareMathOperator{\Ad}{Ad}

\providecommand{\simil}[2]{\operatorname{Sim}\left(#1,#2\right)}
\providecommand{\similb}[3]{\operatorname{Sim}\left(#1,#2,#3\right)}
\providecommand{\reg}[1]{\operatorname{Reg}\left(#1\right)}
\providecommand{\jac}[1]{J_{\mathrm{#1}}}

\providecommand{\ud}{\mathrm{d}}

\newcommand{\ba}{\boldsymbol{a}}
\newcommand{\bA}{\boldsymbol{A}}
\newcommand{\bc}{\boldsymbol{c}}
\newcommand{\be}{\boldsymbol{e}}
\newcommand{\bff}{\boldsymbol{f}}
\newcommand{\bM}{\boldsymbol{M}}
\newcommand{\bp}{\boldsymbol{p}}
\newcommand{\br}{\boldsymbol{r}}
\newcommand{\bs}{\boldsymbol{s}}
\newcommand{\bx}{\boldsymbol{x}}
\newcommand{\by}{\boldsymbol{y}}
\newcommand{\bu}{\boldsymbol{u}}
\newcommand{\bv}{\boldsymbol{v}}
\newcommand{\bw}{\boldsymbol{w}}
\newcommand{\bW}{\boldsymbol{W}}
\newcommand{\balpha}{\boldsymbol{\alpha}}
\newcommand{\bbeta}{\boldsymbol{\beta}}
\newcommand{\bpi}{\boldsymbol{\pi}}
\newcommand{\btheta}{\boldsymbol{\theta}}
\newcommand{\bgamma}{\boldsymbol{\gamma}}
\newcommand{\bnu}{\boldsymbol{\nu}}
\newcommand{\bphi}{\boldsymbol{\phi}}
\newcommand{\bPhi}{\boldsymbol{\Phi}}
\newcommand{\bpsi}{\boldsymbol{\psi}}
\newcommand{\bPsi}{\boldsymbol{\Psi}}
\newcommand{\brho}{\boldsymbol{\rho}}
\newcommand{\bvarphi}{\boldsymbol{\varphi}}
\newcommand{\bvarsigma}{\boldsymbol{\varsigma}}
\newcommand{\btau}{\boldsymbol{\tau}}

\newcommand{\IR}{\mathbb{R}}

% Text commands
\usepackage{xspace}
% \xspace adds space at the end of a macro designed
% for use in text, unless the macro is followed by
% certain punctuation characters

\newcommand{\microns}{\textmu m\xspace}

\newcommand{\cvzr}{Cellvizio\textregistered\xspace}
\newcommand{\cvz}{Cellvizio\xspace}
\newcommand{\FCMLeica}{Leica FCM1000 microscope\xspace}

% Add a period to the end of an abbreviation unless there's one
% already, then \xspace.
\makeatletter
\DeclareRobustCommand\onedot{\futurelet\@let@token\@onedot}
\def\@onedot{\ifx\@let@token.\else.\null\fi\xspace}

\providecommand{\eg}{e.g\onedot}
\providecommand{\ie}{i.e\onedot}
\providecommand{\cf}{cf\onedot}
\providecommand{\etc}{etc\onedot}
\providecommand{\etal}{\emph{et~al}\onedot}

\makeatother


% define custom figref and secref based on eqref
\providecommand{\figref}[1]{Fig.~\ref{#1}}
\providecommand{\Figref}[1]{Figure~\ref{#1}}
\providecommand{\secref}[1]{Section~\ref{#1}}
\providecommand{\Secref}[1]{Section~\ref{#1}}
\providecommand{\chapref}[1]{Chapter~\ref{#1}}
\providecommand{\Chapref}[1]{Chapter~\ref{#1}}

\providecommand{\algref}[1]{Algorithm~\ref{#1}}
\providecommand{\Algref}[1]{Algorithm~\ref{#1}}
\providecommand{\tabref}[1]{Table~\ref{#1}}
\providecommand{\Tabref}[1]{Table~\ref{#1}} 
